%% Generated by Sphinx.
\def\sphinxdocclass{report}
\documentclass[letterpaper,10pt,english]{sphinxmanual}
\ifdefined\pdfpxdimen
   \let\sphinxpxdimen\pdfpxdimen\else\newdimen\sphinxpxdimen
\fi \sphinxpxdimen=.75bp\relax

\PassOptionsToPackage{warn}{textcomp}
\usepackage[utf8]{inputenc}
\ifdefined\DeclareUnicodeCharacter
% support both utf8 and utf8x syntaxes
  \ifdefined\DeclareUnicodeCharacterAsOptional
    \def\sphinxDUC#1{\DeclareUnicodeCharacter{"#1}}
  \else
    \let\sphinxDUC\DeclareUnicodeCharacter
  \fi
  \sphinxDUC{00A0}{\nobreakspace}
  \sphinxDUC{2500}{\sphinxunichar{2500}}
  \sphinxDUC{2502}{\sphinxunichar{2502}}
  \sphinxDUC{2514}{\sphinxunichar{2514}}
  \sphinxDUC{251C}{\sphinxunichar{251C}}
  \sphinxDUC{2572}{\textbackslash}
\fi
\usepackage{cmap}
\usepackage[T1]{fontenc}
\usepackage{amsmath,amssymb,amstext}
\usepackage{babel}



\usepackage{times}
\expandafter\ifx\csname T@LGR\endcsname\relax
\else
% LGR was declared as font encoding
  \substitutefont{LGR}{\rmdefault}{cmr}
  \substitutefont{LGR}{\sfdefault}{cmss}
  \substitutefont{LGR}{\ttdefault}{cmtt}
\fi
\expandafter\ifx\csname T@X2\endcsname\relax
  \expandafter\ifx\csname T@T2A\endcsname\relax
  \else
  % T2A was declared as font encoding
    \substitutefont{T2A}{\rmdefault}{cmr}
    \substitutefont{T2A}{\sfdefault}{cmss}
    \substitutefont{T2A}{\ttdefault}{cmtt}
  \fi
\else
% X2 was declared as font encoding
  \substitutefont{X2}{\rmdefault}{cmr}
  \substitutefont{X2}{\sfdefault}{cmss}
  \substitutefont{X2}{\ttdefault}{cmtt}
\fi


\usepackage[Sonny]{fncychap}
\ChNameVar{\Large\normalfont\sffamily}
\ChTitleVar{\Large\normalfont\sffamily}
\usepackage{sphinx}

\fvset{fontsize=\small}
\usepackage{geometry}


% Include hyperref last.
\usepackage{hyperref}
% Fix anchor placement for figures with captions.
\usepackage{hypcap}% it must be loaded after hyperref.
% Set up styles of URL: it should be placed after hyperref.
\urlstyle{same}
\addto\captionsenglish{\renewcommand{\contentsname}{Contents:}}

\usepackage{sphinxmessages}
\setcounter{tocdepth}{1}



\title{mitiq}
\date{Mar 19, 2020}
\release{0.1.0}
\author{Tech Team @ Unitary Fund}
\newcommand{\sphinxlogo}{\vbox{}}
\renewcommand{\releasename}{Release}
\makeindex
\begin{document}

\ifdefined\shorthandoff
  \ifnum\catcode`\=\string=\active\shorthandoff{=}\fi
  \ifnum\catcode`\"=\active\shorthandoff{"}\fi
\fi

\pagestyle{empty}
\sphinxmaketitle
\pagestyle{plain}
\sphinxtableofcontents
\pagestyle{normal}
\phantomsection\label{\detokenize{index::doc}}



\chapter{Change Log}
\label{\detokenize{changelog:change-log}}\label{\detokenize{changelog:changelog}}\label{\detokenize{changelog::doc}}

\section{Version 0.1.0 (Date)}
\label{\detokenize{changelog:version-0-1-0-date}}\begin{itemize}
\item {} 
\sphinxstylestrong{Initial release.}

\end{itemize}


\chapter{Users Guide}
\label{\detokenize{guide/guide:users-guide}}\label{\detokenize{guide/guide:guide}}\label{\detokenize{guide/guide::doc}}

\section{Overview of mitiq}
\label{\detokenize{guide/guide-overview:overview-of-mitiq}}\label{\detokenize{guide/guide-overview::doc}}
Welcome to \sphinxtitleref{mitiq} Users Guide. The library allows to postprocess results from quantum circuits with both analog and digital techniques, interfacing with a variety of quantum circuit libraries.


\section{Zero Noise Extrapolation}
\label{\detokenize{guide/guide-zne:zero-noise-extrapolation}}\label{\detokenize{guide/guide-zne::doc}}

\subsection{Introduction}
\label{\detokenize{guide/guide-zne:introduction}}
Zero noise extrapolation (ZNE) was introduced concurrently in Ref. {[}1{]} and {[}2{]}.
With \sphinxtitleref{mitiq.zne} module it is possible to extrapolate what the expected value would be without noise. This is done by first setting up one of the key objects in \sphinxtitleref{mitiq}, which is a \sphinxcode{\sphinxupquote{mitiq.Factory}} object.


\subsection{Importing Quantum Circuits}
\label{\detokenize{guide/guide-zne:importing-quantum-circuits}}
\sphinxtitleref{mitiq} allows one to flexibly import and export quantum circuits from other libraries. Here is an example:

\begin{sphinxVerbatim}[commandchars=\\\{\}]
\PYG{g+gp}{\PYGZgt{}\PYGZgt{}\PYGZgt{} }\PYG{k+kn}{from} \PYG{n+nn}{mitiq} \PYG{k+kn}{import} \PYG{n}{Factory}
\end{sphinxVerbatim}


\chapter{Indices and tables}
\label{\detokenize{index:indices-and-tables}}\begin{itemize}
\item {} 
\DUrole{xref,std,std-ref}{genindex}

\item {} 
\DUrole{xref,std,std-ref}{modindex}

\item {} 
\DUrole{xref,std,std-ref}{search}

\end{itemize}



\renewcommand{\indexname}{Index}
\printindex
\end{document}